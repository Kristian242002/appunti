\documentclass[a4paper,12pt]{article}

% Pacchetti utili
\usepackage[utf8]{inputenc}
\usepackage[italian]{babel}
\usepackage{amsmath}
\usepackage{amssymb}
\usepackage{tcolorbox} % per riquadri colorati

% Impostazioni grafiche minime
\usepackage{geometry}
\geometry{margin=2.5cm}

\title{Cos'è l'Informatica}
\author{}
\date{}

\begin{document}

\maketitle

\section{Cos'è l'informatica}
L’informatica è la scienza che si occupa dello \textbf{studio, della rappresentazione e dell’elaborazione automatica delle informazioni}. La parola stessa deriva dall’unione di \emph{informazione} e \emph{automatica}, cioè la capacità di trattare informazioni in modo automatico grazie a strumenti come i computer. Non si limita soltanto all’uso del computer, ma studia anche i principi teorici, i metodi e le tecniche che permettono di risolvere problemi reali attraverso il calcolo e la gestione dei dati.  

Per comprendere meglio il ruolo dell’informatica è fondamentale distinguere due concetti: \textbf{dato} e \textbf{informazione}. Un dato è un elemento grezzo, un fatto isolato o un valore che da solo non ha significato. L’informazione, invece, nasce quando un dato viene interpretato e collocato in un contesto che lo rende comprensibile e utile. Possiamo dire che un dato è come una parola senza spiegazione, mentre l’informazione è la frase che dà senso a quella parola.  

Ad esempio, il numero \texttt{37} preso da solo è un semplice dato, ma se diciamo che “37 °C è la temperatura corporea normale di un essere umano” stiamo fornendo un’informazione. Allo stesso modo, una data come \texttt{2025-10-01} è un dato, ma se aggiungiamo che “il 1° ottobre 2025 si terrà l’interrogazione di informatica” diventa informazione. Il colore \texttt{Verde} è un dato, ma diventa informazione quando sappiamo che “il semaforo è verde, quindi si può attraversare la strada”. Anche un codice come \texttt{101010} è un dato, e assume significato solo se sappiamo che “il numero binario 101010 corrisponde al numero decimale 42”.  

\begin{tcolorbox}[colback=white,colframe=red!80!black,title=Per capire meglio]
Un dato è come un mattone: preso da solo non serve a molto.  
Quando i mattoni vengono messi insieme per costruire una casa, essi diventano informazione.
\end{tcolorbox}

\subsection{Computer}
Il computer si è evoluto nel tempo in molteplici configurazioni, dalle prime macchine da calcolo fino ai moderni dispositivi portatili. Oggi possiamo considerare computer non solo i classici personal computer, ma anche \textbf{notebook}, \textbf{tablet}, \textbf{smartphone} e molti altri dispositivi che utilizziamo quotidianamente. Nonostante le diverse forme, il computer rimane uno \textbf{strumento indispensabile}, il sistema di elaborazione dei dati più diffuso.  

Spesso il computer è chiamato anche \textbf{elaboratore elettronico digitale}.  
\begin{itemize}
    \item È detto \emph{elaboratore} perché è una macchina in grado di trattare dati mediante un programma.  
    \item È detto \emph{elettronico} perché è realizzato con componenti elettronici (circuiti integrati, transistor, microprocessori).  
    \item È detto \emph{digitale} perché elabora e memorizza le informazioni attraverso due simboli fondamentali: \texttt{0} e \texttt{1}, ossia il linguaggio binario.  
\end{itemize}

\begin{tcolorbox}[colback=yellow!10,colframe=yellow!60!black,title=Definizione di Sistema]
Un \textbf{sistema} è un insieme di componenti che interagiscono tra loro per raggiungere un obiettivo comune.  
\end{tcolorbox}

\begin{tcolorbox}[colback=yellow!10,colframe=yellow!60!black,title=Definizione di Programma]
Un \textbf{programma} è una sequenza di istruzioni scritte in un linguaggio comprensibile al computer, che permettono di svolgere un compito preciso.  
\end{tcolorbox}

\begin{tcolorbox}[colback=yellow!10,colframe=yellow!60!black,title=Definizione di Algoritmo]
Un \textbf{algoritmo} è una sequenza finita e ordinata di operazioni che descrive come risolvere un problema.  
\\[6pt]
\textit{Differenza:} un algoritmo è un concetto logico e astratto, mentre un programma è la sua traduzione pratica in un linguaggio che il computer può eseguire.  
\end{tcolorbox}
L'attività svolta da un \textbf{computer} consiste nell'elaborare informazione che provengono dall'esterno e fornire dei risultati.
\begin{itemize}
\item Fase di \textbf{input}: è la fase in cui i dati vengono inseriti nel computer attraverso appositi dispositivi detti \emph{periferiche di input}. 
    In questa fase il computer riceve dall’esterno dati grezzi che non hanno ancora un significato, ma che saranno poi elaborati. 
    \\
    \textit{Esempi di dispositivi di input:} tastiera (inserimento di testo e comandi), mouse o touchpad (puntamento e selezione), microfono (acquisizione della voce), scanner (digitalizzazione di documenti e immagini), fotocamera (acquisizione di foto e video).  
\item Fase di \textbf{elaborazione} : i dati vengono elaborati attraverso un circuito elettronico
    \item Fase di \textbf{output}: è la fase in cui il computer restituisce all’utente i risultati dell’elaborazione dei dati, trasformandoli in informazioni comprensibili. Questo avviene tramite dispositivi detti \emph{periferiche di output}.  
    \\
    \textit{Esempi di dispositivi di output:} monitor o schermo (visualizzazione di testi, immagini e video), stampante (produzione di documenti cartacei), altoparlanti o cuffie (riproduzione di suoni e musica), videoproiettore (proiezione di contenuti su superfici più ampie).  
\end{itemize}
\begin{tcolorbox}[colback=blue!5!white,colframe=blue!80!black,title=Dispositivi di Input/Output]
Esistono anche dispositivi che possono svolgere sia la funzione di \textbf{input} sia di \textbf{output}.  
Ad esempio:  
\begin{itemize}
    \item un \textbf{CD} o un \textbf{DVD}, che possono essere letti (input) o scritti (output);  
    \item un \textbf{hard disk} o una \textbf{chiavetta USB}, che permettono sia la memorizzazione di dati (output) sia la lettura degli stessi (input);  
    \item una \textbf{fotocamera digitale}, che acquisisce immagini (input) ma può anche mostrarle sul display o trasferirle al computer (output).  
\end{itemize}
\end{tcolorbox}
Affinchè il computer possa funzionare,è neccessario che possieda due componenti:l'hardware e il software.
\begin{tcolorbox}[colback=yellow!10,colframe=yellow!60!black,title={Hardware, Software e Firmware}]
\begin{itemize}
    \item \textbf{Hardware}: è la parte fisica e tangibile del computer, composta da tutti i componenti elettronici e meccanici (CPU, memoria, tastiera, monitor, hard disk, ecc.).
    \item \textbf{Software}: è l’insieme dei programmi e delle istruzioni che consentono all’hardware di funzionare e di svolgere compiti specifici (sistemi operativi, applicazioni, giochi, ecc.).
    \item \textbf{Firmware}: è un particolare tipo di software installato direttamente nell’hardware (ad esempio nella scheda madre, in una stampante o in uno smartphone), che ne controlla le funzioni di base e permette l’avvio e la gestione dei dispositivi.
\end{itemize}
\end{tcolorbox}
\subsection{Il Computer e le sue componenti}

Il \textbf{computer} è uno strumento estremamente diffuso e potente non solo perché elabora informazioni, ma anche perché può essere aggiornato e potenziato nel tempo.  
Una delle sue caratteristiche più apprezzate è proprio la possibilità di sostituire o migliorare i componenti interni.  

Tutti i componenti si trovano all’interno di un contenitore chiamato \textbf{case}.  
Esistono diversi tipi di case in base alle dimensioni:
\begin{itemize}
    \item \textbf{Mini tower}: compatto, adatto a spazi ridotti ma con poche possibilità di espansione.
    \item \textbf{Middle tower}: la dimensione più comune, offre un buon equilibrio tra spazio e possibilità di aggiornamento.
    \item \textbf{Big tower}: molto grande, pensato per configurazioni potenti e numerosi componenti aggiuntivi (schede video multiple, sistemi di raffreddamento avanzati, ecc.).
\end{itemize}

\bigskip
\textbf{Componenti principali di un computer:}
\begin{itemize}
    \item \textbf{CPU (Central Processing Unit)}: il cervello del computer, esegue le istruzioni dei programmi e coordina tutte le attività.
    \item \textbf{Scheda madre (Motherboard)}: la scheda principale che collega e permette la comunicazione tra tutti i componenti.
    \item \textbf{RAM (Random Access Memory)}: memoria volatile che contiene temporaneamente i dati e i programmi in uso. Viene svuotata allo spegnimento del computer.
    \item \textbf{Scheda video (VGA / GPU)}: elabora le immagini e i video, fondamentale per giochi, grafica e applicazioni multimediali.
    \item \textbf{HDD (Hard Disk Drive)}: memoria di massa tradizionale con dischi magnetici, grande capacità ma velocità minore.
    \item \textbf{SSD (Solid State Drive)}: memoria di massa moderna basata su chip elettronici, molto più veloce dell’HDD.
    \item \textbf{DVD Writer}: lettore e masterizzatore ottico, oggi meno diffuso ma ancora utilizzato per CD/DVD.
\end{itemize}

\bigskip
\textbf{Differenze importanti:}
\begin{tcolorbox}[colback=yellow!10,colframe=yellow!60!black,title=RAM vs ROM]
\begin{itemize}
    \item \textbf{RAM (Random Access Memory)}: memoria volatile, usata per contenere dati e programmi durante l’esecuzione. Si svuota quando il computer si spegne.
    \item \textbf{ROM (Read Only Memory)}: memoria non volatile, contiene informazioni permanenti (ad esempio il BIOS/firmware). Non si cancella allo spegnimento.
\end{itemize}
\textit{Errore comune:} molti confondono RAM e ROM, ma in realtà hanno funzioni molto diverse.
\end{tcolorbox}

\begin{tcolorbox}[colback=yellow!10,colframe=yellow!60!black,title=RAM vs Hard Disk]
\begin{itemize}
    \item La \textbf{RAM} è veloce ma temporanea: serve per i dati in uso immediato.
    \item L’\textbf{Hard Disk/SSD} è più lento ma permanente: conserva i dati in maniera stabile anche dopo lo spegnimento.
\end{itemize}
\textit{Un esempio semplice:} la RAM è come il tavolo da lavoro dove appoggi i documenti su cui stai lavorando, mentre l’hard disk è l’armadio in cui li archivi per conservarli.
\end{tcolorbox}
\subsection{Fab Lab e Mock-Up}

Un \textbf{Fab Lab} (Fabrication Laboratory) è un laboratorio di \emph{fabbricazione digitale} dove si possono progettare e realizzare oggetti partendo da un modello digitale.  
È uno spazio aperto a studenti, appassionati e professionisti, che offre strumenti moderni per creare prototipi, fare esperimenti e trasformare un’idea in un oggetto reale.  

Tra gli strumenti più comuni presenti in un Fab Lab troviamo:
\begin{itemize}
    \item \textbf{Stampanti 3D}: producono oggetti aggiungendo materiale strato per strato.
    \item \textbf{Fresatrici CNC}: macchine che scolpiscono il materiale rimuovendone una parte (lavorazione sottrattiva).
    \item \textbf{Taglio laser}: permette di incidere o tagliare materiali come legno, plastica o cartone.
    \item \textbf{Scanner 3D}: acquisiscono la forma di un oggetto reale e la trasformano in un modello digitale.
    \item \textbf{Plotter da taglio}: utilizzati per materiali sottili come carta o vinile.
    \item \textbf{Laboratori di elettronica}: con saldatori, schede e sensori per realizzare circuiti.
\end{itemize}

Nel Fab Lab si realizzano spesso dei \textbf{mock-up}, cioè modelli o prototipi di oggetti creati per mostrare come sarà il prodotto finale.  
Un mock-up non serve solo a vedere l’aspetto estetico, ma anche a capire se il progetto funziona, se le proporzioni sono corrette e se deve essere modificato prima della produzione definitiva.  

Grazie alle tecnologie come la stampa 3D o lo scanner 3D, oggi è possibile passare molto velocemente dall’idea al modello fisico, permettendo di testare, correggere e migliorare un progetto in tempi ridotti.  
\subsection{Realtà Virtuale e Realtà Aumentata}

La \textbf{realtà virtuale (VR)} è una tecnologia che permette all’utente di immergersi in un ambiente digitale completamente simulato.  
Indossando dispositivi specifici come i \emph{visori} o gli \textbf{smart glasses}, l’utente si trova circondato da immagini, suoni e a volte anche sensazioni tattili che ricreano uno spazio artificiale, separato dal mondo reale.  
Un esempio di applicazione sono i videogiochi in realtà virtuale, nei quali il giocatore si muove e interagisce dentro un mondo interamente digitale.  

La \textbf{realtà aumentata (AR)}, invece, non sostituisce il mondo reale, ma lo arricchisce sovrapponendo elementi digitali all’ambiente circostante.  
Attraverso la fotocamera di uno smartphone o dispositivi appositi, si possono visualizzare informazioni aggiuntive, animazioni o oggetti virtuali che “si mescolano” con ciò che vediamo dal vivo.  

Tra i primi esempi di applicazioni di realtà aumentata troviamo:
\begin{itemize}
    \item \textbf{Layar}: una delle prime app AR per smartphone, che mostrava contenuti digitali sovrapposti all’inquadratura della fotocamera.
    \item \textbf{Google Tango}: un progetto di Google che consentiva agli smartphone compatibili di mappare l’ambiente in 3D e integrare oggetti virtuali nello spazio reale con grande precisione.
\end{itemize}

In sintesi: la \emph{realtà virtuale} crea un mondo completamente digitale in cui immergersi, mentre la \emph{realtà aumentata} aggiunge contenuti virtuali al mondo reale che ci circonda.  
\newpage
\subsection*{Domande di verifica}

\textbf{Domande aperte}
\begin{enumerate}
    \item Spiega con parole tue la differenza tra \textbf{dato} e \textbf{informazione}, facendo un esempio pratico.
    \item Quali sono le principali differenze tra \textbf{RAM} e \textbf{Hard Disk}?
    \item Descrivi cos’è un \textbf{Fab Lab} e cita almeno due strumenti che si possono trovare al suo interno.
    \item Cosa si intende per \textbf{realtà aumentata} e come si differenzia dalla realtà virtuale?
    \item Elenca e descrivi brevemente le tre fasi principali di funzionamento di un computer (\textit{input}, \textit{elaborazione}, \textit{output}).
\end{enumerate}

\bigskip
\textbf{Domande a crocette}
\begin{enumerate}
    \item[6.] La \textbf{ROM} è:  
    \begin{itemize}
        \item[A)] Una memoria volatile che si cancella allo spegnimento  
        \item[B)] Una memoria non volatile che contiene informazioni permanenti  
        \item[C)] La memoria principale del computer  
        \item[D)] Una periferica di input  
    \end{itemize}

    \item[7.] Quale di questi è un esempio di \textbf{periferica di output}?  
    \begin{itemize}
        \item[A)] Tastiera  
        \item[B)] Microfono  
        \item[C)] Stampante  
        \item[D)] Scanner  
    \end{itemize}

    \item[8.] Quale tra questi dispositivi può essere sia \textbf{input} che \textbf{output}?  
    \begin{itemize}
        \item[A)] Hard disk  
        \item[B)] Tastiera  
        \item[C)] Mouse  
        \item[D)] Monitor  
    \end{itemize}

    \item[9.] La \textbf{CPU} è:  
    \begin{itemize}
        \item[A)] Una memoria permanente del computer  
        \item[B)] Un programma che controlla le periferiche  
        \item[C)] Il processore che esegue le istruzioni e coordina le attività  
        \item[D)] Una scheda grafica  
    \end{itemize}

    \item[10.] Quale tra queste app è stata un esempio di \textbf{realtà aumentata}?  
    \begin{itemize}
        \item[A)] Photoshop  
        \item[B)] Layar  
        \item[C)] Excel  
        \item[D)] WinRAR  
    \end{itemize}
\end{enumerate}
\bigskip

\begin{enumerate}
    \setcounter{enumi}{10} % continua la numerazione da 11
    \item[11.] Quale tra i seguenti è un \textbf{dispositivo di input}?  
    \begin{itemize}
        \item[A)] Monitor  
        \item[B)] Tastiera  
        \item[C)] Stampante  
        \item[D)] Altoparlante  
    \end{itemize}

    \item[12.] Il termine \textbf{firmware} indica:  
    \begin{itemize}
        \item[A)] Un programma temporaneo in RAM  
        \item[B)] Un software che controlla le funzioni di base di un hardware  
        \item[C)] Una memoria esterna di massa  
        \item[D)] Un sistema operativo open source  
    \end{itemize}

    \item[13.] In un computer, la memoria \textbf{cache} serve per:  
    \begin{itemize}
        \item[A)] Archiviare in modo permanente i file dell’utente  
        \item[B)] Fornire uno spazio veloce per i dati usati di frequente dalla CPU  
        \item[C)] Salvare i programmi dopo lo spegnimento  
        \item[D)] Gestire la connessione a Internet  
    \end{itemize}

    \item[14.] La differenza principale tra \textbf{HDD} e \textbf{SSD} è:  
    \begin{itemize}
        \item[A)] L’HDD è più veloce ma costoso, l’SSD è più lento ma economico  
        \item[B)] L’HDD utilizza dischi magnetici, l’SSD chip elettronici  
        \item[C)] L’HDD è una memoria volatile, l’SSD è non volatile  
        \item[D)] L’HDD è usato solo nei server, l’SSD solo nei PC portatili  
    \end{itemize}

    \item[15.] Quale affermazione è \textbf{corretta} riguardo un algoritmo?  
    \begin{itemize}
        \item[A)] È sempre scritto in linguaggio macchina  
        \item[B)] È una sequenza finita e ordinata di operazioni  
        \item[C)] È identico a un programma eseguibile  
        \item[D)] È un componente hardware del computer  
    \end{itemize}
\end{enumerate}

\bigskip
\textbf{Esercizi di completamento}

\begin{enumerate}
    \item[16.] Completa le seguenti frasi inserendo i termini corretti:  
    \begin{itemize}
        \item Il \_\_\_\_\_ è il cervello del computer e si occupa di eseguire le \_\_\_\_\_.  
        \item La \_\_\_\_\_ è una memoria volatile che si svuota allo \_\_\_\_\_.  
        \item La \_\_\_\_\_ è una memoria permanente che contiene istruzioni fisse come il BIOS.  
        \item L’\_\_\_\_\_ è il processo con cui il computer riceve i dati dall’esterno.  
        \item L’\_\_\_\_\_ è la fase in cui i risultati elaborati vengono mostrati all’utente.  
    \end{itemize}

    \item[17.] Completa le frasi con le parole mancanti:  
    \begin{itemize}
        \item Un \_\_\_\_\_ è una sequenza ordinata di operazioni per risolvere un problema.  
        \item Un \_\_\_\_\_ è la sua traduzione in un linguaggio comprensibile al computer.  
        \item Un \_\_\_\_\_ è un insieme di componenti che lavorano per uno scopo comune.  
        \item L’\_\_\_\_\_ è la scheda principale che collega tutti i componenti hardware.  
        \item Un \_\_\_\_\_ è un prototipo creato per testare e visualizzare un progetto prima della produzione definitiva.  
    \end{itemize}
    \item[18.] Completa il seguente testo con i termini corretti:  

    L' \_\_\_\_\_\_\_\_\_\_\_\_ è la parte fisica e tangibile del computer, mentre il \_\_\_\_\_\_\_\_\_\_\_ è l’insieme dei programmi che permettono all’hardware di funzionare.  
    Ogni computer possiede una \_\_\_\_\_\_\_\_\_\_\_, che collega e mette in comunicazione tutti i componenti interni.  
    La memoria \_\_\_\_\_\_\_\_\_\_\_ è detta volatile perché si cancella allo spegnimento, mentre la memoria \_\_\_\_\_\_\_\_\_\_\_ contiene informazioni permanenti.  
    Tra i dispositivi di \_\_\_\_\_\_\_\_\_\_\_ troviamo la tastiera e il mouse, mentre tra quelli di \_\_\_\_\_\_\_\_\_\_\_ ci sono il monitor e la stampante.  
    La \_\_\_\_\_\_\_\_\_\_\_ è il processore che esegue le istruzioni e coordina tutte le attività della macchina.  

    \bigskip
    \textbf{Parole da usare (SONO PIU DEL NECCESSARIO ATTENZIONE!):}  
    CPU, hardware, software, firmware, scheda madre, RAM, ROM, input, output, hard disk, sistema operativo, algoritmo, periferica, mouse, processore

\end{enumerate}

\end{document}
