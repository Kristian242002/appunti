\section{Classi Random e Math}
\subsection{Random}
La libreria designata per usare un oggetto random è \textbf{java.util.Random}.
Per creare un oggetto random bisogna eseguire il seguente commando : \textbf{Random r = new Random();}\\
Vediamo ora altri metodi di questa funzione:
\begin{itemize}
    \item \textbf{boolean nextBoolean()} (genera un numero booleano randomico)
    \item \textbf{double nextDouble()}
    \item \textbf{float nextFloat()}
    \item \textbf{int nextInt()}
    \item \textbf{int nextInt(int max)} (restituisce un numero casuale tra 0 e max escluso)
    \item \textbf{long nextLong()}
\end{itemize}
Vediamo ora un'applicazione pratica:
\begin{lstlisting}[style=java]
// generare un intero randomico e mandarlo a video 
import java.Util.Scanner
import java.Util.Random

public class Main {
    public static void main(String[] args) {
        Random r = new Random();
        int a = r.nextInt();
        System.out.print(a);
    }
}
\end{lstlisting}
\subsection{Math}
La libreria designata per un oggetto Math è \textbf{java.until.Math}\\
Vediamo ora altri metodi di questa funzione:
\begin{itemize}
    \item \textbf{static double E} (costante della classe)
    \item \textbf{static double PI} (costante della classe)
    \item \textbf{static int abs(int i) }(esiste anche per altri tipi numerici)
    \item \textbf{static double cos(double d)}
    \item \textbf{static double log(double d)} (in base e)
    \item \textbf{static double log{10}(double d)} (in base 10)
    \item \textbf{static int max(int a, int b)} (esiste anche per altri tipi numerici)
    \item \textbf{static int min(int a, int b)} (esiste anche per altri tipi numerici)
    \item \textbf{static double sin(double d)}
    \item \textbf{static double sqrt(double d)}
    \item \textbf{static double tan(double d)}
    \item \textbf{static double toDegrees(double radiants)}
    \item \textbf{static double toRadiants(double degrees)}
\end{itemize}
