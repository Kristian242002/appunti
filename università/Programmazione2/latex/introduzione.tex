\section{Introduzione}
Java nasce negli anni 90 dopo C, però il concetto di \textbf{programmazione ad oggetti} esisteva già negli anni 60.
I concetti fondamentali della programmazione ad oggetti sono :
\begin{itemize}
    \item \textbf{Ereditarietà:} è un principio nel quale una classe definita come sottoclasse eredita attributi e metodi fa un'altra classe detta superclasse,cosi da poter riutilizzare il codice e crea una sorta di gerarchia tra le classi. 
    \item \textbf{Incapsulamento:}è il processo per nascondere i dettagli di un oggetto rendendo accessibili solo alcune parti
    \item \textbf{Polimorfismo:}è la capacità di oggetti di diverse classi di rispondere allo stesso metodo in maniera diversa
\end{itemize}
NB. questi concetti veranno ripresi nei capitoli futuri
\subsection{Esecuzione di un programma in Java}
Sappiamo da programmazione I che un file ad esempio pippo.c viene trasformato in eseguibile attraverso il commando \textbf{gcc nomefile.c}.
Facendo così verrà generato un file eseguibile solamente dal sistema operativo nel quale si è generato il file.
Invece con java non è proprio la stessa cosa perchè prendendo in esempio lo stesso file però denominato pippo.java è usando il commando \textbf{javac nomefile.java} verrà formato un file con l'estensione \textbf{.class} esso rappresenta un \textbf{bytecode}.
I file .class possono essere eseguiti solo da un interprete Java,esso è un vantaggio perchè si potra eseguire in ogni sistema operativo \textbf{solamente} se si conterrà un interprete java.
Da qua in poi ogni linguaggio di programmazione utilizzerà questo concetto per poter eseguire i file.
NB.il file java deve sempre contenere la prima lettera maiuscola.

\begin{lstlisting}[style=java]
public class HelloWorld {
    public static void main(String[] args) {
        // Stampa "Hello, World!" nella console
        System.out.println("Hello, World!");
    }
}
\end{lstlisting}
Ora generiamo l'eseguibile:
\begin{lstlisting}[style=command]
    >>javac Hello.java  //generera il file Hello.class
    >>java Hello        //esegue il file
    >>Hello, World!
\end{lstlisting}
In questo corso useremo \textbf{Eclipse for java developers}.
\subsection{I tipi di variabili}
I tipi primitivi presenti in Java sono 8:
\begin{enumerate}
    \item boolean (o True o False)
    \item char (singolo carratere)
    \item byte (8 bit)
    \item short (16 bit)
    \item int (32 bit)
    \item long (64 bit)
    \item float (rappresenta numeri in virgola mobile) (32)
    \item double (rappresenta numeri in virgola mobile) (64)
\end{enumerate}
Inoltre esistono i tipi \textbf{var} i quali rappresentano un qualsiasi tipo di variabile e vengono identificati proprio da java.
Per rappresentare una costante bisogna scrivere \textbf{final tipo nome}.
\begin{lstlisting}[style=java]
public class variabili {
    public static void main(String[] args) {
        final int contatore = 0; // costante
        var variabile ;// java assegnera il tipo 
    }
}
\end{lstlisting}
NB. di solito si preferisce non usare il tipo var per la leggibilità del codice.